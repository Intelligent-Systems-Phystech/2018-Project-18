\documentclass[12pt,twoside]{article}
\usepackage{jmlda}

\begin{document}

\title
    {Прогнозирование намерений. Построение оптимальной модели декодирования сигналов при моделировании нейрокомпьютерного интерфейса.}
\author
    {Г.\,Р.~Латыпова} 
\email
    {latypova.gr@phystech.edu}
\organization
    {МФТИ (ГУ)}
\abstract
    {Нейрокомпьютерный интерфейс позволяет помочь людям с ограниченными возможностями вернуть их мобильность. По имеющемуся описанию сигнала прибора можно смоделировать поведение субъекта. В данной работе построена единая система, решающая задачу декодирования сигналов. В качестве этапов построения такой системы были решены задачи предобработки данных, выделения признакового пространства, снижения размерности и выбора модели оптимальной сложности. В работе учитывается комплексная природа сигнала: непрерывная траектория движения, наличие дискретных структурных переменных, наличие непрерывных переменных.}

\titleEng
    [Short title]
    {Prediction of intentions. Building an optimal model for decoding signals when modeling a brain-computer interface.}
\authorEng
    {G.\,R.~Latypova}
\organizationEng
    {Moscow Institute of Physics and Technology}
\abstractEng
    {The brain-computer interface allows people with disabilities to regain their mobility. Having a description of the signal of the device, it is possible to simulate the behavior of the subject. In this paper, a unified system is constructed that solves the problem of decoding signals. As stages of building such a system, the tasks of data preprocessing, selection of attribute space, reduction of dimensionality and choice of a model of optimal complexity were solved. The work takes into account the complex nature of the signal: continuous trajectory of motion, the presence of discrete structural variables, the presence of continuous variables.}

\maketitle

\end{document}
