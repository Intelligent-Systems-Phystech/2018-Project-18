\documentclass[12pt,twoside]{article}
\usepackage{jmlda}

\begin{document}

\title
    {Прогнозирование намерений. Построение оптимальной модели декодирования сигналов при моделировании нейрокомпьютерного интерфейса.}
\author
    {Н.\,Д.~Суходольский} 
\email
    {suhodolskiy.nd@phystech.edu}
\organization
    {МФТИ (ГУ)}
\abstract
    {В данной работе по имеющемуся описанию ECoG сигнала прибора моделируется поведение субъекта. В работе предлагается построить систему, решающую задачу декодирования сигналов. В качестве этапов построения такой системы решаются задачи предобработки данных, выделения признакового пространства и снижения размерности. Помимо этого, внимание уделяется выбору модели оптимальной сложности. В работе учитывается комплексная природа сигнала: непрерывная траектория движения, наличие дискретных структурных переменных, наличие непрерывных переменных.}

\maketitle
\linenumbers

\section{Введение}
    {Нейрокомпьютерный интерфейс (BCI) осуществляет обмен информацией между мозгом и электронным устройством. Одним из применений этой технологии является помощь людям с ограниченными возможностями в возвращении их мобильности.
    
    При решении подобных задач данные мозга часто получают используя электро- и магнитоэнцефалографию (EEG и MEG соответственно) \cite{Waldert2008}. В данной работе используются данные, полученые с помощью электрокортикографии (ECoG), которая позволяет получить измерения с большим разрешением во временной и пространственной областях. Имеются работы, в которых ECoG применяется для определения движения, выполняемого субъектом, например, движения руки \cite{Chao2010}.

Построение модели также требует снижения размерности признакового пространства. В данной работе используются метод наименьщих частных квадратов (Partial Least Squares) \cite{Krishnan2011}, используемый в современных исследованиях по этой тематике \cite{Eliseyev2011} \cite{Eliseyev2012} и метод главных компонент (Principal Component Analysis) \cite{Brems2017}

В построенной модели учитывается комплексная природа сигнала. Приняты во внимание непрерывная траектория движения, наличие дискретных структурных переменных и непрерывных переменных. Исследование проводится на основе данных из библиотеки сигналов человеческого мозга ECoG \cite{Data2016}. 
}
\bibliographystyle{plain}
\bibliography{Sukhodolskiy2018Project18}

\end{document}